\section{El diccionario álgebra-geometría}

\begin{preliminaries}
Sea $V \subset \A^n$ un conjunto algebraico afín.
\end{preliminaries}

\begin{proposition}
Los subconjuntos algebraicos de $V$ forman una familia cerrada bajo uniones finitas e intersecciones arbitrarias. Por ende, sus complementos forman una topología sobre $V$.
\end{proposition}

\begin{proof}
Ver \cite[p. 2]{hartshorne}.
\end{proof}

\begin{definition}
La \textit{topología de Zariski} sobre $V$ es la topología de la proposición anterior.
\end{definition}

\begin{remark}
Por el teorema de la base de Hilbert, $V$ es un espacio topológico noetheriano.
\end{remark}

\begin{proposition}
Un conjunto algebraico afín es irreducible si y sólo si su ideal es primo.
\end{proposition}

\begin{proof}
Ver \cite[p. 15]{fulton}.
\end{proof}

\begin{corollary}
Las componentes irreducibles de $V$ son los conjuntos $V(\p)$, donde $\p \subset k[\A^n]$ es un ideal primo minimal entre aquellos que contienen a $I(V)$. \qed
\end{corollary}

\begin{theorem}
(Teorema de los ceros de Hilbert)

\begin{itemize}
    \item Todo ideal maximal de $k[\A^n]$ es el ideal de un punto $p \in \A^n$.
    \item Todo ideal propio de $k[\A^n]$ se anula en algún punto $p \in \A^n$.
    \item Sea $\a \subset k[\A^n]$ un ideal arbitrario. Entonces $I(V(\a)) = \Rad(\a)$.
\end{itemize}
\end{theorem}

\begin{proof}
Ver \cite[pp. 20-22]{fulton}.
\end{proof}

\begin{corollary}
Existen correspondencias biyectivas entre

\begin{itemize}
    \item Los puntos de $\A^n$ y los ideales maximales de $k[\A^n]$.
    \item Los subconjuntos algebraicos irreducibles de $\A^n$ y los ideales primos de $k[\A^n]$.
    \item Los subconjuntos algebraicos de $\A^n$ y los ideales radicales de $k[\A^n]$. \qed
\end{itemize}
\end{corollary}
