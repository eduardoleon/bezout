\chapter{Variedades afines}

\noindent La geometría algebraica estudia las \textit{variedades algebraicas}, espacios conformados por los ceros de una familia de polinomios. Los dos tipos más importantes de variedades algebraicas, y los únicos que discutiremos en este trabajo, son las \textit{variedades afines} y las \textit{variedades proyectivas}.

En este capítulo, presentamos los hechos básicos sobre las variedades afines que utilizaremos en el resto del trabajo. El más importante de ellos es que toda variedad afín es caracterizada de manera precisa por un \textit{anillo de coordenadas}. Esta caracterización se extiende a una equivalencia dual entre la categoría de variedades afines y la categoría de $k$-álgebras finitamente generadas sin elementos divisores de cero.

Fijemos a lo largo de este trabajo un cuerpo algebraicamente cerrado $k$.
