\section{El diccionario álgebra-geometría}

\begin{preliminaries}
Sea $V \subset \P^n$ un conjunto algebraico proyectivo.
\end{preliminaries}

\begin{proposition}
Los subconjuntos algebraicos de $V$ forman una familia cerrada bajo uniones finitas e intersecciones arbitrarias. Por ende, sus complementos forman una topología sobre $V$.
\end{proposition}

\begin{proof}
Idéntica a la proposición 1.5.
\end{proof}

\begin{definition}
La \textit{topología de Zariski} sobre $V$ es la topología de la proposición anterior.
\end{definition}

\begin{remark}
Por el teorema de la base de Hilbert, $V$ es un espacio topológico noetheriano.
\end{remark}

\begin{definition}
El \textit{ideal irrelevante} $\m \subset k_h[\P^n]$ es el ideal del conjunto proyectivo vacío.
\end{definition}

\begin{proposition}
Un conjunto algebraico proyectivo es irreducible si y sólo si su ideal es primo, pero no irrelevante.
\end{proposition}

\begin{proof}
Idéntica a la proposición 1.6.
\end{proof}

\begin{corollary}
Las componentes irreducibles de $V$ son los conjuntos $V(\p)$, donde $\p \subset k_h[\P^n]$ es un ideal primo minimal entre aquellos que contienen a $I(V)$ y no son irrelevantes. \qed
\end{corollary}

\begin{corollary}
El conjunto $V$ tiene dimensión $\dim V = \dim C(V) - 1$. \qed
\end{corollary}

\begin{corollary}
Existen correspondencias biyectivas entre

\begin{itemize}
    \item Los subconjuntos algebraicos de $\P^n$ y los ideales radicales homogéneos propios de $k_h[\P^n]$.
    \item Los subconjuntos algebraicos irreducibles de $\P^n$ y los ideales primos homogéneos de $k_h[\P^n]$, distintos del ideal irrelevante. \qed
\end{itemize}
\end{corollary}
