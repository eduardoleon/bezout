\section{Variedades casi proyectivas}

\begin{preliminaries}
Sean $V \subset \P^n$ y $W \subset \P^m$ dos variedades casi proyectivas, tal y como definimos a continuación.
\end{preliminaries}

\begin{definition}
Una \textit{variedad casi proyectiva} es un subconjunto abierto de una variedad proyectiva.
\end{definition}

\begin{definition}
Un \textit{morfismo regular} es una función continua $\varphi : V \to W$ tal que, para toda función regular $f : U \to k$ definida sobre un abierto de Zariski $U \subset W$, el pullback $\varphi^\star(f) : \varphi^{-1}(U) \to k$ es también una función regular.
\end{definition}

\begin{definition}
Un \textit{isomorfismo regular} es un morfismo regular invertible cuya inversa es regular.
\end{definition}

\begin{proposition}
Sea $U \subset \P^n$ el complemento de un hiperplano. Entonces existe un isomorfismo regular $\varphi : \A^n \to U$.
\end{proposition}

\begin{proof}
Supongamos sin pérdida de generalidad que $U$ es el complemento del hiperplano $x_0 = 0$. Entonces $\varphi(x_1 \dots x_n) = [1 : x_1 : \dots : x_n]$ es un isomorfismo.
\end{proof}

\begin{definition}
Un \textit{morfismo racional} $\varphi : V \dashrightarrow W$ es un morfismo regular $\varphi : U \to W$ que parte de un subconjunto abierto denso $U \subset V$. Decimos que $\varphi$ está \textit{definido} sobre los puntos $p \in U$.
\end{definition}

\begin{definition}
Un \textit{morfismo birracional} $\varphi : V \dashrightarrow V'$ es un isomorfismo regular $\varphi : U \to U'$ entre subconjuntos abiertos y densos de $V, V'$, respectivamente.
\end{definition}

\begin{example}
El espacio afín $\A^n$ es birracionalmente equivalente al espacio proyectivo $\P^n$.
\end{example}

\begin{proposition}
Dos variedades son \textit{birracionalmente equivalentes} si y sólo si sus cuerpos de fracciones son isomorfos.
\end{proposition}

\begin{proof}
Ver \cite[p. 155]{fulton}.
\end{proof}
