\section{Automorfismos de $\P^n$}

\noindent Identifiquemos el espacio afín $\A^{n+1}$ con el espacio vectorial $k^{n+1}$ y consideremos la acción de $k^\star$ sobre $\A^{n+1}$ vía reescalamientos. El espacio de órbitas de esta acción no es muy bien comportado, ya que el origen está en la vecindad de todas las demás órbitas\footnote{En el lenguaje moderno, decimos que este espacio de órbitas es un esquema no separado.}. Sin embargo, si removemos este punto fijo problemático, recuperamos el espacio proyectivo $\P^n$.

Es evidente que todo automorfismo $\tilde \varphi : \A^{n+1} \to \A^{n+1}$ que respeta las órbitas de $k^\star$ induce un automorfismo $\varphi : \P^n \to \P^n$. En este caso, tenemos $\varphi \circ L = L \circ \tilde \varphi$, donde $L$ envía cada $p \in \A^{n+1}$ a su órbita $L_p \in \P^n$. Por ende, el cociente $\GL(\A^{n+1}) / k^\star$ es un subgrupo de $\Aut(\P^n)$.

La prueba de la inclusión reversa se dará en la sección 6.4.
