\section{Anillos y módulos graduados asociados}

\noindent Las referencias para esta sección son \cite[pp. 111-112]{atiyah} y \cite[pp. 146-148]{eisenbud}.

\begin{preliminaries}
Fijemos un anillo noetheriano $R$, un ideal $\a \subset R$ y un $R$-módulo $M$ finitamente generado equipado con una $\a$-filtración estable.
\end{preliminaries}

\begin{definition}
El \textit{anillo graduado asociado} de $R$ es el anillo graduado
$$G(R) = \dfrac R \a \oplus \dfrac \a {\a^2} \oplus \dfrac {\a^2} {\a^3} \oplus \dfrac {\a^3} {\a^4} \oplus \dots$$
\end{definition}

\begin{remark}
Equivalentemente, podemos definir $G(R) = R^\star / \a R^\star$.
\end{remark}

\begin{definition}
El \textit{módulo graduado asociado} de $M$ es el $G(R)$-módulo graduado
$$G(M) = \dfrac {M_0} {M_1} \oplus \dfrac {M_1} {M_2} \oplus \dfrac {M_2} {M_3} \oplus \dots$$
\end{definition}

\begin{proposition}
El anillo graduado asociado $G(R)$ es noetheriano.
\end{proposition}

\begin{proof}
El anillo $G(R) = R^\star / \a R^\star$ es noetheriano porque $R^\star$ es noetheriano.
\end{proof}

\begin{proposition}
El módulo graduado asociado $G(M)$ es finitamente generado.
\end{proposition}

\begin{proof}
Sea $r \in \N$ el instante en que $M_n$ se estabiliza. Para todo $n \ge r$, tenemos
$$\dfrac {\a} {\a^2} \dfrac {M_n} {M_{n+1}} = \dfrac {M_{n+1}} {M_{n+2}}$$

Por ende, $G(M)$ es generado por el prefijo
$$\tilde M = \dfrac {M_0} {M_1} \oplus \dots \oplus \dfrac {M_r} {M_{r+1}}$$

Las partes homogéneas $M_n / M_{n+1}$ son $R/\a$-módulos finitamente generados, por ende $\tilde M$ es un $R/\a$-módulo finitamente generado, por ende $G(M)$ es un $G(R)$-módulo finitamente generado.
\end{proof}
