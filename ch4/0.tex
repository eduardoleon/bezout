\chapter{Anillos y módulos filtrados}

\noindent Una manera de simplificar el estudio de una variedad algebraica es limitar nuestra atención a la \textit{vecindad infinitesimal} de un punto. Esta vecindad infinitesimal no es realmente un subconjunto abierto de la variedad original, sino un objeto abstracto\footnote{En la teoría de esquemas, una vecindad infinitesimal es un objeto geométrico por cuenta propia. Sin embargo, el estudio de los esquemas trasciende el alcance de este trabajo.} que captura las propiedades geométricas de la variedad en el punto de interés.

Intuitivamente, uno podría pensar que la vecindad infinitesimal de $p \in V$ es representada por el anillo local $\O_p(V)$, pero esto no es el caso\footnote{El anillo local $\O_p(V)$ describe una vecindad genérica de Zariski de $p \in V$. Pero las vecindades de Zariski distan mucho de ser infinitesimales. En la geometría diferencial, donde los espacios son de Hausdorff y los puntos tienen vecindades arbitrariamente pequeñas, los anillos locales reflejan mejor la idea de vecindad infinitesimal.}. De hecho, es fácil ver que $\O_p(V)$ retiene demasiada información global sobre $V$: si $\O_p(V)$ y $\O_q(W)$ son isomorfos, entonces $k(V)$ y $k(W)$ también son isomorfos, por ende las variedades $V$ y $W$ son birracionalmente equivalentes.

Supongamos que $U$ es una vecindad infinitesimal y $f : U \to k$ es una ``función regular'' sobre ella. La propiedad fundamental de $f$ es el orden con que se anula en el único punto $p \in U$. No es necesario que $f$ se pueda extender a una vecindad finita de $U$.

Para ver una vecindad infinitesimal, necesitamos un microscopio algebraico que magnifique la región cercana al punto de interés, en un sentido más preciso que el indicado por la topología de Zariski. En este capítulo construiremos el microscopio.
