\chapter*{Resumen}

\noindent El teorema de Bézout afirma que dos curvas polinomiales ``suficientemente genéricas'' $f(x,y) = 0$ y $g(x,y) = 0$ se intersecan en $\deg f \cdot \deg g$ puntos, contados con multiplicidad y considerando los puntos de intersección ``en el infinito'', si los hubiera. Este trabajo tiene por objetivo demostrar una generalización del teorema de Bézout aplicable a la intersección de variedades en el espacio proyectivo de dimensión arbitraria.

Como punto de partida, asumimos que el lector está familiarizado con las estructuras básicas del álgebra conmutativa (anillos, ideales, módulos) y un mínimo de topología general. El álgebra conmutativa tiene un rol fundamental en la geometría algebraica, ya que es el lenguaje formal en que las ideas geométricas intuitivas se traducen en argumentos matemáticos rigurosos y de gran generalidad. Comparativamente, la topología juega un rol secundario, consolidando definiciones y proposiciones que no dependen demasiado de los detalles algebraicos.

Nuestra estrategia para estudiar las intersecciones de variedades proyectivas es asignar a cada variedad proyectiva $Y \subset \P^n$ un polinomio $P_Y \in \Q[n]$, llamado el \textit{polinomio de Hilbert} de $Y$, que encapsula la información aditiva de $Y$ en un objeto fácil de manipular. La intersección de $Y$ con una hipersuperficie transversal en $\P^n$ da lugar a una sucesión exacta que involucra a $P_Y$, a partir de la cual, por un argumento más tedioso que difícil, se desprende la generalización buscada del teorema de Bézout.
