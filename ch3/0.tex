\chapter{Anillos y módulos graduados}

\noindent La relación entre una variedad proyectiva y su anillo de coordenadas homogéneo es más delicada y sutil que su contraparte en el caso afín. Para empezar, los elementos del anillo no son funciones regulares sobre la variedad\footnote{De hecho, las únicas funciones regulares definidas sobre toda una variedad proyectiva son las constantes.}. Dado un elemento homogéneo del anillo, tiene sentido preguntar en qué puntos se anula dicho elemento. Sin embargo, los elementos no homogéneos no tienen ningún significado geométrico.

Para enfatizar que sólo nos interesan los elementos homogéneos de un anillo o un módulo, lo equiparemos con una \textit{gradación}. En este capítulo, desarrollaremos las herramientas técnicas para extraer la mayor cantidad de información posible sobre un módulo graduado. La más importante de estas herramientas es el \textit{polinomio de Hilbert}, que ocupará un lugar central en este trabajo.
