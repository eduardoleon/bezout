\section{Polinomio de Hilbert}

\noindent La referencia para esta sección es \cite[pp. 117-118]{atiyah}.

\begin{preliminaries}
Sean $R$ un anillo graduado noetheriano y $M$ un $R$-módulo graduado finitamente generado, como en la sección 3.2. Supongamos que los generadores $x_i$ de $R$ son de grado $d_i = 1$. Sea $\lambda$ una función aditiva no negativa sobre la clase de $R_0$-módulos finitamente generados.
\end{preliminaries}

\begin{proposition}
Existe un polinomio $P_M \in \Q[n]$ de grado $r = \dim M$ tal que $\lambda(M_n) = P_M(n)$ para todo $n \gg 0$.
\end{proposition}

\begin{proof}
Puesto que todos los $x_i$ son de grado $1$, el denominador de $S_M$ es una potencia de $1-t$. Simplificando todas las ocurrencias del factor $1-t$ en el numerador, tenemos
$$S_M(t) = \frac {g(t)} {(1-t)^{r+1}}$$

Expresando los factores como series de potencias, tenemos
$$\frac 1 {(1-t)^{r+1}} = \sum_{n=0}^\infty \binom {n+r} r t^n, \qquad g(t) = \sum_{j=a}^b u_j t^j$$

Entonces, para todo $n \ge b$, el coeficiente correspondiente en la serie de Poincaré es
$$\lambda(M_n) = P_M(n) = \sum_{j=a}^b u_j \binom {n+r-j} r$$

Por simple inspección, $P_M$ es un polinomio numérico de grado $r$.
\end{proof}

\begin{definition}
El polinomio $P_M = [c_0; \dots; c_r]$ de la proposición anterior se denomina \textit{polinomio de Hilbert} de $M$ con respecto a $\lambda$. El coeficiente $c_0$ es el \textit{grado} de $M$.
\end{definition}

\begin{corollary}
Sea $x \in R$ un elemento homogéneo que no es divisor de cero en $M$. Si $M$ tiene dimensión positiva, entonces $N = M/xM$ tiene grado $\deg N = \deg M \cdot \deg x$.
\end{corollary}

\begin{proof}
Sea $d = \deg x$. Puesto que el polinomio de Hilbert es aditivo,
$$P_N(n) = P_M(n) - P_{M(-d)}(n) = [c_0; \dots; c_r](n) - [c_0; \dots; c_r](n-d) = [dc_0; \dots](n)$$

Entonces $N$ tiene grado $\deg N = dc_0 = \deg M \cdot \deg x$.
\end{proof}
