\section{Definiciones básicas}

\noindent La referencia para esta sección es \cite[p. 52]{hartshorne}.

\begin{preliminaries}
En este capítulo, $R = k[x_0 \dots x_n]$ es el anillo de coordenadas homogéneo de $\P^n$.
\end{preliminaries}

\begin{notation}
En este capítulo, denotaremos el ideal de $Y \subset \P^n$ por $I_Y$, en vez de $I(Y)$.
\end{notation}

\begin{definition}
El \textit{polinomio de Hilbert} de $Y \subset \P^n$, denotado $P_Y \in \Q[n]$, es el polinomio de Hilbert de su anillo de coordenadas homogéneo $R_Y = R/I_Y$ con respecto a la función $\lambda = \dim_k$.
\end{definition}

\begin{remark}
Por construcción, $\dim Y = \dim R_Y$.
\end{remark}

\begin{definition}
El \textit{grado} de $Y \subset \P^n$ es el grado de $R_Y$ como $R$-módulo.
\end{definition}

\begin{remark}
El grado de $Y \subset \P^n$ no es una propiedad de $Y$ como variedad abstracta, sino de un encaje $\iota : Y \to \P^n$. Por ejemplo, una recta proyectiva $L \subset \P^2$ y una cónica proyectiva $C \subset \P^2$ son isomorfas como variedades abstractas, pero $\deg L = 1$, mientras que $\deg C = 2$.
\end{remark}

\begin{example}
Toda variedad no vacía $Y \subset \P^n$ tiene grado positivo.
\end{example}

\begin{example}
El espacio proyectivo $\P^n$ tiene grado $\deg \P^n = \deg R = 1$.
\end{example}

\begin{remark}
Por el argumento de estrellas y barras, $P_R = [1; 0; \dots; 0]$.
\end{remark}

\begin{example}
Sean $H \subset \P^n$ una hipersuperficie y $f \in R$ un generador de $I_H \subset R$. Entonces $H$ tiene grado $\deg H = \deg f$.
\end{example}

\begin{remark}
Por el corolario 3.9, $\deg H = \deg R_H = \deg {(R / fR)} = \deg R \cdot \deg f = \deg f$.
\end{remark}
